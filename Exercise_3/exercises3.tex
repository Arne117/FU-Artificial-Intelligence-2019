\documentclass[a4paper,twoside,12pt]{article}
\usepackage{amsfonts,amstext}
\usepackage{amsmath}
\usepackage{amsthm}
\usepackage{amssymb}
\usepackage{german}
\usepackage{graphicx}
\usepackage{fullpage}
\usepackage[utf8]{inputenc}
\usepackage{hyperref}
\usepackage{algorithm}
\usepackage{enumerate}
\usepackage{enumitem}

\newcommand{\ZETTELNUMMER}{3}

\newcounter{AUFGNR}
\setcounter{AUFGNR}{6}
\newcommand{\AUFGABE}[1]{\vspace{0.3cm}\item[Exercise~{\arabic{AUFGNR}}:]\stepcounter{AUFGNR} #1\\}

\begin{document}
\pagestyle{plain}
\noindent Abdullah Barhoum, Linus Helfmann, Arne Rolf, Linus Ververs

\hrule\medskip
\rule{0ex}{0ex}\\[-1ex]
Lösungen zur Exercise \ZETTELNUMMER

\smallskip
\noindent
\large
\textbf{Künstliche Intelligenz}
\medskip\hrule

\smallskip
\noindent
\begin{enumerate}[leftmargin=2.5cm]
	\AUFGABE{Truth Tables}
	%p,q,r,qvr,qp,rp,a+b+c,a+b+c->-q
	\\[-2ex]
	\hspace*{-2.5cm}
	\begin{tabular}{|c|c|c|c|c|c|c|c|}
		\hline 
		p & q & r & $q \vee p$ & $q \Rightarrow \neg p$ & $\neg(r \wedge p)$  & $\{...\}$ & $\{...\} \Rightarrow \neg p$  \\ 
		\hline 
		0 & 0 & 0 & 0 & 1 & 1 & 0 & 1 \\ 
		\hline 
		0 & 0 & 1 & 0 & 1 & 1 & 0 & 1 \\ 
		\hline 
		0 & 1 & 0 & 1 & 1 & 1 & 1 & 1 \\ 
		\hline 
		0 & 1 & 1 & 1 & 1 & 1 & 1 & 1 \\ 
		\hline 
		1 & 0 & 0 & 1 & 1 & 1 & 1 & 0 \\ 
		\hline 
		1 & 0 & 1 & 1 & 1 & 0 & 0 & 1 \\ 
		\hline 
		1 & 1 & 0 & 1 & 0 & 1 & 0 & 1 \\ 
		\hline 
		1 & 1 & 1 & 1 & 0 & 0 & 0 & 1 \\ 
		\hline 
	\end{tabular} 
	%	Use a truth table to prove that ¬p is a logical consequence of the set {q ∨ r, q ⇒ ¬p, ¬(r ∧ p)}.
	\AUFGABE{SLD}
%	Give SLD-resolution refutations for the following sets of clauses:
%	• {P1}, {P2}, {P3}, {P4}, {¬P1, ¬P2, P6}, {¬P3, ¬P4, P7}, {¬P6, ¬P7, P8}, {¬P8}
%	• {¬P2, P3}, {¬P3, P4}, {¬P4, P5}, {P3}, {P1}, {P2}, {¬P1}, {¬P3, P6}, {¬P3, P7},	{¬P3, P8}
	\AUFGABE{DPLL I}
	%	Are the following formulas satisfiable? Use the DPLL procedure:
	%(a) (¬a ∨ b) ∧ (¬c ∨ d) ∧ (¬e ∨ ¬f) ∧ (f ∨ ¬e ∨ ¬b)
%	(b) (p ∨ q ∨ r ∨ s) ∧ (¬p ∨ q ∨ ¬r) ∧ (¬q ∨ ¬r ∨ s) ∧ (p ∨ ¬q ∨ r ∨ s) ∧ (q ∨ ¬r ∨ ¬s) ∧	(¬p ∨ ¬r ∨ s) ∧ (¬p ∨ ¬s) ∧ (p ∨ ¬q)
	\AUFGABE{DPLL II}
	% Can you present a formula that well illustrates the worst case
	\AUFGABE{Pythagoreans Triple Problem}
	% Find out and explain how the pythagoreans triple problem was represented in SAT by Heule and colleagues.
\end{enumerate}

\end{document}
