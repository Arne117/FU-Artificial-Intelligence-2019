\documentclass[a4paper,twoside,12pt]{article}
\usepackage{amsfonts,amstext}
\usepackage{amsmath}
\usepackage{amsthm}
\usepackage{amssymb}
\usepackage{german}
\usepackage{graphicx}
\usepackage{fullpage}
\usepackage[utf8]{inputenc}
\usepackage{hyperref}
\usepackage{algorithm}
\usepackage{enumerate}
\usepackage{enumitem}

\newcommand{\ZETTELNUMMER}{3}

\newcounter{AUFGNR}
\setcounter{AUFGNR}{6}
\newcommand{\AUFGABE}[1]{\vspace{0.3cm}\item[Exercise~{\arabic{AUFGNR}}:]\stepcounter{AUFGNR} #1\\[1ex]}
\setlength{\itemindent}{-2cm}

\begin{document}
\pagestyle{plain}
\noindent Abdullah Barhoum, Linus Helfmann, Arne Rolf, Linus Ververs

\hrule\medskip
\rule{0ex}{0ex}\\[-1ex]
Lösungen zur Exercise \ZETTELNUMMER

\smallskip
\noindent
\large
\textbf{Künstliche Intelligenz}
\medskip\hrule

\smallskip
\noindent
%[align=parleft, labelsep=0.5cm, wide=-0.5cm]
\begin{enumerate}[wide=-0.2cm]
	\AUFGABE{Truth Tables}
	Use a truth table to prove that $\neg p$ is a logical consequence of the set ${q \vee r, q \Rightarrow \neg p, \neg (r \wedge p)}$.\\
	\begin{tabular}{|c|c|c|c|c|c|c|c|}
		\hline 
		p & q & r & $q \vee p$ & $q \Rightarrow \neg p$ & $\neg(r \wedge p)$  & $\{...\}$ & $\{...\} \Rightarrow \neg p$  \\ 
		\hline 
		0 & 0 & 0 & 0 & 1 & 1 & 0 & 1 \\ 
		\hline 
		0 & 0 & 1 & 0 & 1 & 1 & 0 & 1 \\ 
		\hline 
		0 & 1 & 0 & 1 & 1 & 1 & 1 & 1 \\ 
		\hline 
		0 & 1 & 1 & 1 & 1 & 1 & 1 & 1 \\ 
		\hline 
		1 & 0 & 0 & 1 & 1 & 1 & 1 & 0 \\ 
		\hline 
		1 & 0 & 1 & 1 & 1 & 0 & 0 & 1 \\ 
		\hline 
		1 & 1 & 0 & 1 & 0 & 1 & 0 & 1 \\ 
		\hline 
		1 & 1 & 1 & 1 & 0 & 0 & 0 & 1 \\ 
		\hline 
	\end{tabular} 
	\AUFGABE{SLD}
	Give SLD-resolution refutations for the following sets of clauses:\\
	a) $\{P1\}, \{P2\}, \{P3\}, \{P4\}, \{\neg P1, \neg P2, P6\}, \{\neg P3, \neg P4, P7\}, \\
	~~~~~ \{\neg P6, \neg P7, P8\}, \{\neg P8\}$
	\begin{align*}
		\underline{\{\neg P8\}, \{\neg P6, \neg P7, P8\}} \hspace{11.65cm}\\
		\underline{\{\neg P6, \neg P7\}, \{\neg P1, \neg P2, P6\}} \hspace{7.9cm} \\
		\underline{\{\neg P1,\neg P2, \neg P7\}, \{P2\}} \hspace{6.5cm} \\
		\underline{\{\neg P1, \neg P7\},\{P1\}} \hspace{5.1cm}\\
		\underline{\{\neg P7\}, \{\neg P3, \neg P4, P7\}} \hspace{1.35cm} \\
	 	\underline{\{\neg P3, \neg P4\}, \{P3\}} \\
	 	\underline{\{\neg P4\}, \{P4\}} \\
	 \{\}
	\end{align*}
	b) $\{\neg P2, P3\}, \{\neg P3, P4\}, \{\neg P4, P5\}, \{P3\}, \{P1\}, \{P2\}, \{\neg P1\},\\
	~~~~~ \{\neg P3, P6\}, \{\neg P3, P7\},	\{\neg P3, P8\}$
	\begin{align*}
		\underline{\{\neg P1\}, \{P1\}} \\
		\{\}
	\end{align*}
	\AUFGABE{DPLL I}
	Are the following formulas satisfiable? Use the DPLL procedure:\\
	a) $(\neg  a \vee b) \wedge (\neg c \vee d) \wedge (\neg e \vee \neg f) \wedge (f \vee \neg e \vee \neg b)$\\
	\begin{align*} 
		[&]& (\neg  a \vee b) \wedge (\neg c \vee d) \wedge (\neg e \vee \neg f) \wedge (f \vee \neg e \vee \neg b)  &&|&~\text{Pure-Literal } \neg a\\
		[&\neg a]& (\neg c \vee d) \wedge (\neg e \vee \neg f) \wedge (f \vee \neg e \vee \neg b) &&|&~\text{Pure-Literal } \neg c\\
		[&\neg a, \neg c]& (\neg e \vee \neg f) \wedge (f \vee \neg e \vee \neg b) &&|&~\text{Split } \neg e\\
		[&\neg a, \neg c, \neg c]& \emptyset &&&\\
	\end{align*}

	b) $(p \vee q \vee r \vee s) \wedge (\neg p \vee q \vee \neg r) \wedge (\neg q \vee \neg r \vee s) \wedge (p \vee \neg q \vee r \vee s) \wedge (q \vee \neg r \vee \neg s) \wedge	(\neg p \vee \neg r \vee s) \wedge (\neg p \vee \neg s) \wedge (p \vee \neg q)$ Split $s$
	\begin{align*} 
	[&s]& (\neg p \vee q \vee \neg r) \wedge (q \vee \neg r )  \wedge (\neg p) \wedge (p \vee \neg q) &&|& \text{Unitprop } \neg p\\
	[&s, \neg p]& (q \vee \neg r ) \wedge (\neg q) &&|& \text{Unitprop } \neg q\\
	[&s, \neg p, \neg q]& ( \neg r ) &&|& \text{Unitprop } \neg r\\
	[&s, \neg p, \neg q, \neg r]& \emptyset\\
	\end{align*}
	
	\AUFGABE{DPLL II}
	Can you present a formula that well illustrates the worst case.\\
	\\
	Der Worst Case, ist wenn man nie PureLiteral/Unit-Propagation anwenden kann, sondern nur Split mit Backtrack.\\
	Für n Variablen, erstelle alle Teilformeln der Länge n wo alle Variablen vorkommen mit allen kombinationen von und/oder. Dadurch gibt es k.....
	\AUFGABE{Pythagoreans Triple Problem}
	Find out and explain how the pythagoreans triple problem was represented in SAT by Heule and colleagues.\\
	\\
	\(F_n\) repräsentiert das Problem die Zahlen \(1...n\) so mit 2 Farben einzufärben, dass es kein einfarbiges Pythagoreisches Tripel gibt. Um dieses Problem in SAT abzubilden, werden die Variablen \(x_i \in  \{1...n\}\), die durch ihre Belegung (True/False) repräsentieren, wie die entsprechende Zahl \(i\) eingefärbt wird. Heißt: True entspricht Farbe 1 und False Farbe 2.  Für jedes Pthagoreisches Tripel \((a,b,c)\) mit \(a^2 + b^2  = c^2\) ergeben sich folgende Klauseln: 
	\[(x_a \vee x_b \vee x_c) \wedge (\neg  x_a \vee \neg x_b \vee \neg x_c)\] All diese Klauseln zusammen ergeben das zu \(F_n\) äquivalente SAT-Problem.
	
\end{enumerate}

\end{document}
